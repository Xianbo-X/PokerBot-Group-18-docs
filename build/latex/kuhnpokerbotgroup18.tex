%% Generated by Sphinx.
\def\sphinxdocclass{report}
\documentclass[letterpaper,10pt,english]{sphinxmanual}
\ifdefined\pdfpxdimen
   \let\sphinxpxdimen\pdfpxdimen\else\newdimen\sphinxpxdimen
\fi \sphinxpxdimen=.75bp\relax
\ifdefined\pdfimageresolution
    \pdfimageresolution= \numexpr \dimexpr1in\relax/\sphinxpxdimen\relax
\fi
%% let collapsible pdf bookmarks panel have high depth per default
\PassOptionsToPackage{bookmarksdepth=5}{hyperref}

\PassOptionsToPackage{warn}{textcomp}
\usepackage[utf8]{inputenc}
\ifdefined\DeclareUnicodeCharacter
% support both utf8 and utf8x syntaxes
  \ifdefined\DeclareUnicodeCharacterAsOptional
    \def\sphinxDUC#1{\DeclareUnicodeCharacter{"#1}}
  \else
    \let\sphinxDUC\DeclareUnicodeCharacter
  \fi
  \sphinxDUC{00A0}{\nobreakspace}
  \sphinxDUC{2500}{\sphinxunichar{2500}}
  \sphinxDUC{2502}{\sphinxunichar{2502}}
  \sphinxDUC{2514}{\sphinxunichar{2514}}
  \sphinxDUC{251C}{\sphinxunichar{251C}}
  \sphinxDUC{2572}{\textbackslash}
\fi
\usepackage{cmap}
\usepackage[T1]{fontenc}
\usepackage{amsmath,amssymb,amstext}
\usepackage{babel}



\usepackage{tgtermes}
\usepackage{tgheros}
\renewcommand{\ttdefault}{txtt}



\usepackage[Bjarne]{fncychap}
\usepackage{sphinx}

\fvset{fontsize=auto}
\usepackage{geometry}


% Include hyperref last.
\usepackage{hyperref}
% Fix anchor placement for figures with captions.
\usepackage{hypcap}% it must be loaded after hyperref.
% Set up styles of URL: it should be placed after hyperref.
\urlstyle{same}

\addto\captionsenglish{\renewcommand{\contentsname}{Contents:}}

\usepackage{sphinxmessages}
\setcounter{tocdepth}{1}



\title{Kuhn PokerBot Group 18}
\date{Apr 06, 2022}
\release{v1.2.0}
\author{Xianbo XU, Yao Lu, Bairui Han, Wen\sphinxhyphen{}Hung Huang}
\newcommand{\sphinxlogo}{\vbox{}}
\renewcommand{\releasename}{Release}
\makeindex
\begin{document}

\pagestyle{empty}
\sphinxmaketitle
\pagestyle{plain}
\sphinxtableofcontents
\pagestyle{normal}
\phantomsection\label{\detokenize{index::doc}}
\sphinxAtStartPar
Welcome to Kuhn PokerBot Group 18’s documentation!


\bigskip\hrule\bigskip


\sphinxAtStartPar
This documentation is for APIs of the PokerBot of Group 18

\sphinxstepscope



\sphinxAtStartPar
\sphinxstylestrong{Kuhn poker} is an extremely simplified form of poker developed by Harold W. Kuhn as a simple model zero\sphinxhyphen{}sum two\sphinxhyphen{}player imperfect\sphinxhyphen{}information game, amenable to a complete game\sphinxhyphen{}theoretic analysis. In Kuhn poker, the deck includes only three playing cards, for example a King, Queen, and Jack. One card is dealt to each player, which may place bets similarly to a standard poker.
If both players bet or both players pass, the player with the higher card wins, otherwise, the betting player wins. \sphinxcite{Introduction:kuhnpoker}

\sphinxstepscope


\chapter{agent module}
\label{\detokenize{agent:agent-module}}\label{\detokenize{agent::doc}}
\sphinxAtStartPar
For agent class, we treat image classifiers and action generators as two components of the agent. So, two member variables \sphinxtitleref{self.model} and \sphinxtitleref{self.strategy} are introduced to provide services for the agent.
By designing like this, our agent class will focus on how to pass information from image classifiers to the strategy generator. And the agent is responsible for passing information between internal components image classifiers, strategy generator and external controller.
Also, we can easily use new image classifiers and new strategy generators to replace current components without having to consider the implementation of the agent. We just make sure the two components follow interface specifications and change the loaded \sphinxtitleref{self.model\textasciigrave{}} and \sphinxtitleref{self.strategy}.
Then our agent can run normally without any further changes in the agent class.

\phantomsection\label{\detokenize{agent:module-agent}}\index{module@\spxentry{module}!agent@\spxentry{agent}}\index{agent@\spxentry{agent}!module@\spxentry{module}}\index{PokerAgent (class in agent)@\spxentry{PokerAgent}\spxextra{class in agent}}

\begin{fulllineitems}
\phantomsection\label{\detokenize{agent:agent.PokerAgent}}
\pysigstartsignatures
\pysigline{\sphinxbfcode{\sphinxupquote{class\DUrole{w}{  }}}\sphinxcode{\sphinxupquote{agent.}}\sphinxbfcode{\sphinxupquote{PokerAgent}}}
\pysigstopsignatures
\sphinxAtStartPar
Bases: \sphinxcode{\sphinxupquote{object}}
\index{make\_action() (agent.PokerAgent method)@\spxentry{make\_action()}\spxextra{agent.PokerAgent method}}

\begin{fulllineitems}
\phantomsection\label{\detokenize{agent:agent.PokerAgent.make_action}}
\pysigstartsignatures
\pysiglinewithargsret{\sphinxbfcode{\sphinxupquote{make\_action}}}{\emph{\DUrole{n}{state}\DUrole{p}{:}\DUrole{w}{  }\DUrole{n}{{\hyperref[\detokenize{client:client.state.ClientGameState}]{\sphinxcrossref{client.state.ClientGameState}}}}}, \emph{\DUrole{n}{round}\DUrole{p}{:}\DUrole{w}{  }\DUrole{n}{{\hyperref[\detokenize{client:client.state.ClientGameRoundState}]{\sphinxcrossref{client.state.ClientGameRoundState}}}}}}{{ $\rightarrow$ str}}
\pysigstopsignatures
\sphinxAtStartPar
Next action, used to choose a new action depending on the current state of the game. This method implements your
unique PokerBot strategy. Use the state and round arguments to decide your next best move.
\begin{quote}\begin{description}
\item[{Parameters}] \leavevmode\begin{itemize}
\item {} 
\sphinxAtStartPar
\sphinxstyleliteralstrong{\sphinxupquote{state}} ({\hyperref[\detokenize{client:client.state.ClientGameState}]{\sphinxcrossref{\sphinxstyleliteralemphasis{\sphinxupquote{ClientGameState}}}}}) \textendash{} State object of the current game (a game has multiple rounds)

\item {} 
\sphinxAtStartPar
\sphinxstyleliteralstrong{\sphinxupquote{round}} ({\hyperref[\detokenize{client:client.state.ClientGameRoundState}]{\sphinxcrossref{\sphinxstyleliteralemphasis{\sphinxupquote{ClientGameRoundState}}}}}) \textendash{} State object of the current round (from deal to showdown)

\end{itemize}

\item[{Returns}] \leavevmode
\sphinxAtStartPar
A string representation of the next action an agent wants to do next, should be from a list of available actions

\item[{Return type}] \leavevmode
\sphinxAtStartPar
str in {[}‘BET’, ‘CALL’, ‘CHECK’, ‘FOLD’{]} (and in round.get\_available\_actions())

\end{description}\end{quote}

\end{fulllineitems}

\index{on\_error() (agent.PokerAgent method)@\spxentry{on\_error()}\spxextra{agent.PokerAgent method}}

\begin{fulllineitems}
\phantomsection\label{\detokenize{agent:agent.PokerAgent.on_error}}
\pysigstartsignatures
\pysiglinewithargsret{\sphinxbfcode{\sphinxupquote{on\_error}}}{\emph{\DUrole{n}{error}}}{}
\pysigstopsignatures
\sphinxAtStartPar
This methods will be called in case of error either from server backend or from client itself. You can
optionally use this function for error handling.
\begin{quote}\begin{description}
\item[{Parameters}] \leavevmode
\sphinxAtStartPar
\sphinxstyleliteralstrong{\sphinxupquote{error}} (\sphinxstyleliteralemphasis{\sphinxupquote{str}}) \textendash{} string representation of the error

\end{description}\end{quote}

\end{fulllineitems}

\index{on\_game\_end() (agent.PokerAgent method)@\spxentry{on\_game\_end()}\spxextra{agent.PokerAgent method}}

\begin{fulllineitems}
\phantomsection\label{\detokenize{agent:agent.PokerAgent.on_game_end}}
\pysigstartsignatures
\pysiglinewithargsret{\sphinxbfcode{\sphinxupquote{on\_game\_end}}}{\emph{\DUrole{n}{state}\DUrole{p}{:}\DUrole{w}{  }\DUrole{n}{{\hyperref[\detokenize{client:client.state.ClientGameState}]{\sphinxcrossref{client.state.ClientGameState}}}}}, \emph{\DUrole{n}{result}\DUrole{p}{:}\DUrole{w}{  }\DUrole{n}{str}}}{}
\pysigstopsignatures
\sphinxAtStartPar
This method is called once after the game has ended. A game ends automatically. You can optionally use this
method for logging purposes.
\begin{quote}\begin{description}
\item[{Parameters}] \leavevmode\begin{itemize}
\item {} 
\sphinxAtStartPar
\sphinxstyleliteralstrong{\sphinxupquote{state}} ({\hyperref[\detokenize{client:client.state.ClientGameState}]{\sphinxcrossref{\sphinxstyleliteralemphasis{\sphinxupquote{ClientGameState}}}}}) \textendash{} State object of the current game

\item {} 
\sphinxAtStartPar
\sphinxstyleliteralstrong{\sphinxupquote{result}} (\sphinxstyleliteralemphasis{\sphinxupquote{str in}}\sphinxstyleliteralemphasis{\sphinxupquote{ {[}}}\sphinxstyleliteralemphasis{\sphinxupquote{\textquotesingle{}WIN\textquotesingle{}}}\sphinxstyleliteralemphasis{\sphinxupquote{, }}\sphinxstyleliteralemphasis{\sphinxupquote{\textquotesingle{}DEFEAT\textquotesingle{}}}\sphinxstyleliteralemphasis{\sphinxupquote{{]}}}) \textendash{} End result of the game

\end{itemize}

\end{description}\end{quote}

\end{fulllineitems}

\index{on\_game\_start() (agent.PokerAgent method)@\spxentry{on\_game\_start()}\spxextra{agent.PokerAgent method}}

\begin{fulllineitems}
\phantomsection\label{\detokenize{agent:agent.PokerAgent.on_game_start}}
\pysigstartsignatures
\pysiglinewithargsret{\sphinxbfcode{\sphinxupquote{on\_game\_start}}}{\emph{\DUrole{n}{gametype}\DUrole{p}{:}\DUrole{w}{  }\DUrole{n}{str}}}{}
\pysigstopsignatures
\sphinxAtStartPar
This method will be called once at the beginning of the game when server confirms both players have connected.

\end{fulllineitems}

\index{on\_image() (agent.PokerAgent method)@\spxentry{on\_image()}\spxextra{agent.PokerAgent method}}

\begin{fulllineitems}
\phantomsection\label{\detokenize{agent:agent.PokerAgent.on_image}}
\pysigstartsignatures
\pysiglinewithargsret{\sphinxbfcode{\sphinxupquote{on\_image}}}{\emph{\DUrole{n}{image}}}{}
\pysigstopsignatures
\sphinxAtStartPar
This method is called every time when card image changes. Use this method for image recongition procedure.
\begin{quote}\begin{description}
\item[{Parameters}] \leavevmode
\sphinxAtStartPar
\sphinxstyleliteralstrong{\sphinxupquote{image}} (\sphinxstyleliteralemphasis{\sphinxupquote{Image}}) \textendash{} Image object

\end{description}\end{quote}

\end{fulllineitems}

\index{on\_new\_round\_request() (agent.PokerAgent method)@\spxentry{on\_new\_round\_request()}\spxextra{agent.PokerAgent method}}

\begin{fulllineitems}
\phantomsection\label{\detokenize{agent:agent.PokerAgent.on_new_round_request}}
\pysigstartsignatures
\pysiglinewithargsret{\sphinxbfcode{\sphinxupquote{on\_new\_round\_request}}}{\emph{\DUrole{n}{state}\DUrole{p}{:}\DUrole{w}{  }\DUrole{n}{{\hyperref[\detokenize{client:client.state.ClientGameState}]{\sphinxcrossref{client.state.ClientGameState}}}}}}{}
\pysigstopsignatures
\sphinxAtStartPar
This method is called every time before a new round is started. A new round is started automatically.
You can optionally use this method for logging purposes.
\begin{quote}\begin{description}
\item[{Parameters}] \leavevmode
\sphinxAtStartPar
\sphinxstyleliteralstrong{\sphinxupquote{state}} ({\hyperref[\detokenize{client:client.state.ClientGameState}]{\sphinxcrossref{\sphinxstyleliteralemphasis{\sphinxupquote{ClientGameState}}}}}) \textendash{} State object of the current game

\end{description}\end{quote}

\end{fulllineitems}

\index{on\_round\_end() (agent.PokerAgent method)@\spxentry{on\_round\_end()}\spxextra{agent.PokerAgent method}}

\begin{fulllineitems}
\phantomsection\label{\detokenize{agent:agent.PokerAgent.on_round_end}}
\pysigstartsignatures
\pysiglinewithargsret{\sphinxbfcode{\sphinxupquote{on\_round\_end}}}{\emph{\DUrole{n}{state}\DUrole{p}{:}\DUrole{w}{  }\DUrole{n}{{\hyperref[\detokenize{client:client.state.ClientGameState}]{\sphinxcrossref{client.state.ClientGameState}}}}}, \emph{\DUrole{n}{round}\DUrole{p}{:}\DUrole{w}{  }\DUrole{n}{{\hyperref[\detokenize{client:client.state.ClientGameRoundState}]{\sphinxcrossref{client.state.ClientGameRoundState}}}}}}{}
\pysigstopsignatures
\sphinxAtStartPar
This method is called every time a round has ended. A round ends automatically. You can optionally use this
method for logging purposes.
\begin{quote}\begin{description}
\item[{Parameters}] \leavevmode\begin{itemize}
\item {} 
\sphinxAtStartPar
\sphinxstyleliteralstrong{\sphinxupquote{state}} ({\hyperref[\detokenize{client:client.state.ClientGameState}]{\sphinxcrossref{\sphinxstyleliteralemphasis{\sphinxupquote{ClientGameState}}}}}) \textendash{} State object of the current game

\item {} 
\sphinxAtStartPar
\sphinxstyleliteralstrong{\sphinxupquote{round}} ({\hyperref[\detokenize{client:client.state.ClientGameRoundState}]{\sphinxcrossref{\sphinxstyleliteralemphasis{\sphinxupquote{ClientGameRoundState}}}}}) \textendash{} State object of the current round

\end{itemize}

\end{description}\end{quote}

\end{fulllineitems}


\end{fulllineitems}


\sphinxstepscope


\chapter{models package}
\label{\detokenize{models:models-package}}\label{\detokenize{models::doc}}

\section{Model handling:}
\label{\detokenize{models:model-handling}}\begin{quote}

\sphinxAtStartPar
As for model part, we hoped our model to be robust for different noise images so we generated different noise level images with big rotation and put all these images to training set. Our models would be able to learn many kinds of noisy images and finally the model had great accuracy on even very high noise. The CNN model performed better than FCN and we built both models with 3 and 4 classes outputs.
\end{quote}


\section{Submodules}
\label{\detokenize{models:submodules}}

\section{models.CNN3 module}
\label{\detokenize{models:module-models.CNN3}}\label{\detokenize{models:models-cnn3-module}}\index{module@\spxentry{module}!models.CNN3@\spxentry{models.CNN3}}\index{models.CNN3@\spxentry{models.CNN3}!module@\spxentry{module}}\index{model (class in models.CNN3)@\spxentry{model}\spxextra{class in models.CNN3}}

\begin{fulllineitems}
\phantomsection\label{\detokenize{models:models.CNN3.model}}
\pysigstartsignatures
\pysigline{\sphinxbfcode{\sphinxupquote{class\DUrole{w}{  }}}\sphinxcode{\sphinxupquote{models.CNN3.}}\sphinxbfcode{\sphinxupquote{model}}}
\pysigstopsignatures
\sphinxAtStartPar
Bases: {\hyperref[\detokenize{models:models.base_model.PokerModelBase}]{\sphinxcrossref{\sphinxcode{\sphinxupquote{models.base\_model.PokerModelBase}}}}}

\sphinxAtStartPar
Model of Image classifier
PokerModelBase provide the interfaces definition for image classifiers.
model define output with 3 classes using CNN
\index{forward() (models.CNN3.model method)@\spxentry{forward()}\spxextra{models.CNN3.model method}}

\begin{fulllineitems}
\phantomsection\label{\detokenize{models:models.CNN3.model.forward}}
\pysigstartsignatures
\pysiglinewithargsret{\sphinxbfcode{\sphinxupquote{forward}}}{\emph{\DUrole{n}{x}}}{}
\pysigstopsignatures\begin{quote}\begin{description}
\item[{Parameters}] \leavevmode
\sphinxAtStartPar
\sphinxstyleliteralstrong{\sphinxupquote{x}} (\sphinxstyleliteralemphasis{\sphinxupquote{tensor}}) \textendash{} Input tensor for the neural network

\item[{Returns}] \leavevmode
\sphinxAtStartPar
\sphinxstylestrong{l3} \textendash{} Output tensor with score for classes

\item[{Return type}] \leavevmode
\sphinxAtStartPar
tensor

\end{description}\end{quote}

\end{fulllineitems}

\index{predict() (models.CNN3.model method)@\spxentry{predict()}\spxextra{models.CNN3.model method}}

\begin{fulllineitems}
\phantomsection\label{\detokenize{models:models.CNN3.model.predict}}
\pysigstartsignatures
\pysiglinewithargsret{\sphinxbfcode{\sphinxupquote{predict}}}{\emph{\DUrole{n}{x}}}{}
\pysigstopsignatures\begin{quote}\begin{description}
\item[{Parameters}] \leavevmode
\sphinxAtStartPar
\sphinxstyleliteralstrong{\sphinxupquote{x}} (\sphinxstyleliteralemphasis{\sphinxupquote{tensor}}) \textendash{} Input tensor for the neural network

\item[{Returns}] \leavevmode
\sphinxAtStartPar
\sphinxstylestrong{labels} \textendash{} The predicted labels are returned in shape {[}n\_batches{]}.

\item[{Return type}] \leavevmode
\sphinxAtStartPar
torch.tensor

\end{description}\end{quote}

\end{fulllineitems}

\index{prob() (models.CNN3.model method)@\spxentry{prob()}\spxextra{models.CNN3.model method}}

\begin{fulllineitems}
\phantomsection\label{\detokenize{models:models.CNN3.model.prob}}
\pysigstartsignatures
\pysiglinewithargsret{\sphinxbfcode{\sphinxupquote{prob}}}{\emph{\DUrole{n}{x}}}{}
\pysigstopsignatures\begin{quote}\begin{description}
\item[{Parameters}] \leavevmode
\sphinxAtStartPar
\sphinxstyleliteralstrong{\sphinxupquote{x}} (\sphinxstyleliteralemphasis{\sphinxupquote{tensor}}) \textendash{} Input tensor for the neural network

\item[{Returns}] \leavevmode
\sphinxAtStartPar
The predicted probabilities for each class are returned in shape {[}n\_batches,n\_classes{]}.

\item[{Return type}] \leavevmode
\sphinxAtStartPar
torch.tensor

\end{description}\end{quote}

\end{fulllineitems}

\index{training (models.CNN3.model attribute)@\spxentry{training}\spxextra{models.CNN3.model attribute}}

\begin{fulllineitems}
\phantomsection\label{\detokenize{models:models.CNN3.model.training}}
\pysigstartsignatures
\pysigline{\sphinxbfcode{\sphinxupquote{training}}\sphinxbfcode{\sphinxupquote{\DUrole{p}{:}\DUrole{w}{  }bool}}}
\pysigstopsignatures
\end{fulllineitems}


\end{fulllineitems}



\section{models.CNN4 module}
\label{\detokenize{models:module-models.CNN4}}\label{\detokenize{models:models-cnn4-module}}\index{module@\spxentry{module}!models.CNN4@\spxentry{models.CNN4}}\index{models.CNN4@\spxentry{models.CNN4}!module@\spxentry{module}}\index{model (class in models.CNN4)@\spxentry{model}\spxextra{class in models.CNN4}}

\begin{fulllineitems}
\phantomsection\label{\detokenize{models:models.CNN4.model}}
\pysigstartsignatures
\pysigline{\sphinxbfcode{\sphinxupquote{class\DUrole{w}{  }}}\sphinxcode{\sphinxupquote{models.CNN4.}}\sphinxbfcode{\sphinxupquote{model}}}
\pysigstopsignatures
\sphinxAtStartPar
Bases: {\hyperref[\detokenize{models:models.base_model.PokerModelBase}]{\sphinxcrossref{\sphinxcode{\sphinxupquote{models.base\_model.PokerModelBase}}}}}

\sphinxAtStartPar
Model of Image classifier
PokerModelBase provide the interfaces definition for image classifiers.
model define output with 4 classes using CNN
\index{forward() (models.CNN4.model method)@\spxentry{forward()}\spxextra{models.CNN4.model method}}

\begin{fulllineitems}
\phantomsection\label{\detokenize{models:models.CNN4.model.forward}}
\pysigstartsignatures
\pysiglinewithargsret{\sphinxbfcode{\sphinxupquote{forward}}}{\emph{\DUrole{n}{x}}}{}
\pysigstopsignatures\begin{quote}\begin{description}
\item[{Parameters}] \leavevmode
\sphinxAtStartPar
\sphinxstyleliteralstrong{\sphinxupquote{x}} (\sphinxstyleliteralemphasis{\sphinxupquote{tensor}}) \textendash{} Input tensor for the neural network

\item[{Returns}] \leavevmode
\sphinxAtStartPar
\sphinxstylestrong{l3} \textendash{} Output tensor with score for classes

\item[{Return type}] \leavevmode
\sphinxAtStartPar
tensor

\end{description}\end{quote}

\end{fulllineitems}

\index{predict() (models.CNN4.model method)@\spxentry{predict()}\spxextra{models.CNN4.model method}}

\begin{fulllineitems}
\phantomsection\label{\detokenize{models:models.CNN4.model.predict}}
\pysigstartsignatures
\pysiglinewithargsret{\sphinxbfcode{\sphinxupquote{predict}}}{\emph{\DUrole{n}{x}}}{}
\pysigstopsignatures\begin{quote}\begin{description}
\item[{Parameters}] \leavevmode
\sphinxAtStartPar
\sphinxstyleliteralstrong{\sphinxupquote{x}} (\sphinxstyleliteralemphasis{\sphinxupquote{tensor}}) \textendash{} Input tensor for the neural network

\item[{Returns}] \leavevmode
\sphinxAtStartPar
\sphinxstylestrong{labels} \textendash{} The predicted labels are returned in shape {[}n\_batches{]}.

\item[{Return type}] \leavevmode
\sphinxAtStartPar
torch.tensor

\end{description}\end{quote}

\end{fulllineitems}

\index{prob() (models.CNN4.model method)@\spxentry{prob()}\spxextra{models.CNN4.model method}}

\begin{fulllineitems}
\phantomsection\label{\detokenize{models:models.CNN4.model.prob}}
\pysigstartsignatures
\pysiglinewithargsret{\sphinxbfcode{\sphinxupquote{prob}}}{\emph{\DUrole{n}{x}}}{}
\pysigstopsignatures\begin{quote}\begin{description}
\item[{Parameters}] \leavevmode
\sphinxAtStartPar
\sphinxstyleliteralstrong{\sphinxupquote{x}} (\sphinxstyleliteralemphasis{\sphinxupquote{tensor}}) \textendash{} Input tensor for the neural network

\item[{Returns}] \leavevmode
\sphinxAtStartPar
The predicted probabilities for each class are returned in shape {[}n\_batches,n\_classes{]}.

\item[{Return type}] \leavevmode
\sphinxAtStartPar
torch.tensor

\end{description}\end{quote}

\end{fulllineitems}

\index{training (models.CNN4.model attribute)@\spxentry{training}\spxextra{models.CNN4.model attribute}}

\begin{fulllineitems}
\phantomsection\label{\detokenize{models:models.CNN4.model.training}}
\pysigstartsignatures
\pysigline{\sphinxbfcode{\sphinxupquote{training}}\sphinxbfcode{\sphinxupquote{\DUrole{p}{:}\DUrole{w}{  }bool}}}
\pysigstopsignatures
\end{fulllineitems}


\end{fulllineitems}



\section{models.FCN3 module}
\label{\detokenize{models:module-models.FCN3}}\label{\detokenize{models:models-fcn3-module}}\index{module@\spxentry{module}!models.FCN3@\spxentry{models.FCN3}}\index{models.FCN3@\spxentry{models.FCN3}!module@\spxentry{module}}\index{model (class in models.FCN3)@\spxentry{model}\spxextra{class in models.FCN3}}

\begin{fulllineitems}
\phantomsection\label{\detokenize{models:models.FCN3.model}}
\pysigstartsignatures
\pysigline{\sphinxbfcode{\sphinxupquote{class\DUrole{w}{  }}}\sphinxcode{\sphinxupquote{models.FCN3.}}\sphinxbfcode{\sphinxupquote{model}}}
\pysigstopsignatures
\sphinxAtStartPar
Bases: {\hyperref[\detokenize{models:models.base_model.PokerModelBase}]{\sphinxcrossref{\sphinxcode{\sphinxupquote{models.base\_model.PokerModelBase}}}}}

\sphinxAtStartPar
Model of Image classifier
PokerModelBase provide the interfaces definition for image classifiers.
model define output with 3 classes using FCN
\index{forward() (models.FCN3.model method)@\spxentry{forward()}\spxextra{models.FCN3.model method}}

\begin{fulllineitems}
\phantomsection\label{\detokenize{models:models.FCN3.model.forward}}
\pysigstartsignatures
\pysiglinewithargsret{\sphinxbfcode{\sphinxupquote{forward}}}{\emph{\DUrole{n}{x}}}{}
\pysigstopsignatures\begin{quote}\begin{description}
\item[{Parameters}] \leavevmode
\sphinxAtStartPar
\sphinxstyleliteralstrong{\sphinxupquote{x}} (\sphinxstyleliteralemphasis{\sphinxupquote{tensor}}) \textendash{} Input tensor for the neural network

\item[{Returns}] \leavevmode
\sphinxAtStartPar
\sphinxstylestrong{l3} \textendash{} Output tensor with score for classes

\item[{Return type}] \leavevmode
\sphinxAtStartPar
tensor

\end{description}\end{quote}

\end{fulllineitems}

\index{predict() (models.FCN3.model method)@\spxentry{predict()}\spxextra{models.FCN3.model method}}

\begin{fulllineitems}
\phantomsection\label{\detokenize{models:models.FCN3.model.predict}}
\pysigstartsignatures
\pysiglinewithargsret{\sphinxbfcode{\sphinxupquote{predict}}}{\emph{\DUrole{n}{x}}}{}
\pysigstopsignatures\begin{quote}\begin{description}
\item[{Parameters}] \leavevmode
\sphinxAtStartPar
\sphinxstyleliteralstrong{\sphinxupquote{x}} (\sphinxstyleliteralemphasis{\sphinxupquote{tensor}}) \textendash{} Input tensor for the neural network

\item[{Returns}] \leavevmode
\sphinxAtStartPar
\sphinxstylestrong{labels} \textendash{} The predicted labels are returned in shape {[}n\_batches{]}.

\item[{Return type}] \leavevmode
\sphinxAtStartPar
torch.tensor

\end{description}\end{quote}

\end{fulllineitems}

\index{prob() (models.FCN3.model method)@\spxentry{prob()}\spxextra{models.FCN3.model method}}

\begin{fulllineitems}
\phantomsection\label{\detokenize{models:models.FCN3.model.prob}}
\pysigstartsignatures
\pysiglinewithargsret{\sphinxbfcode{\sphinxupquote{prob}}}{\emph{\DUrole{n}{x}}}{}
\pysigstopsignatures\begin{quote}\begin{description}
\item[{Parameters}] \leavevmode
\sphinxAtStartPar
\sphinxstyleliteralstrong{\sphinxupquote{x}} (\sphinxstyleliteralemphasis{\sphinxupquote{tensor}}) \textendash{} Input tensor for the neural network

\item[{Returns}] \leavevmode
\sphinxAtStartPar
The predicted probabilities for each class are returned in shape {[}n\_batches,n\_classes{]}.

\item[{Return type}] \leavevmode
\sphinxAtStartPar
torch.tensor

\end{description}\end{quote}

\end{fulllineitems}

\index{training (models.FCN3.model attribute)@\spxentry{training}\spxextra{models.FCN3.model attribute}}

\begin{fulllineitems}
\phantomsection\label{\detokenize{models:models.FCN3.model.training}}
\pysigstartsignatures
\pysigline{\sphinxbfcode{\sphinxupquote{training}}\sphinxbfcode{\sphinxupquote{\DUrole{p}{:}\DUrole{w}{  }bool}}}
\pysigstopsignatures
\end{fulllineitems}


\end{fulllineitems}



\section{models.FCN4 module}
\label{\detokenize{models:module-models.FCN4}}\label{\detokenize{models:models-fcn4-module}}\index{module@\spxentry{module}!models.FCN4@\spxentry{models.FCN4}}\index{models.FCN4@\spxentry{models.FCN4}!module@\spxentry{module}}\index{model (class in models.FCN4)@\spxentry{model}\spxextra{class in models.FCN4}}

\begin{fulllineitems}
\phantomsection\label{\detokenize{models:models.FCN4.model}}
\pysigstartsignatures
\pysigline{\sphinxbfcode{\sphinxupquote{class\DUrole{w}{  }}}\sphinxcode{\sphinxupquote{models.FCN4.}}\sphinxbfcode{\sphinxupquote{model}}}
\pysigstopsignatures
\sphinxAtStartPar
Bases: {\hyperref[\detokenize{models:models.base_model.PokerModelBase}]{\sphinxcrossref{\sphinxcode{\sphinxupquote{models.base\_model.PokerModelBase}}}}}

\sphinxAtStartPar
Model of Image classifier
PokerModelBase provide the interfaces definition for image classifiers.
model define output with 4 classes using FCN
\index{forward() (models.FCN4.model method)@\spxentry{forward()}\spxextra{models.FCN4.model method}}

\begin{fulllineitems}
\phantomsection\label{\detokenize{models:models.FCN4.model.forward}}
\pysigstartsignatures
\pysiglinewithargsret{\sphinxbfcode{\sphinxupquote{forward}}}{\emph{\DUrole{n}{x}}}{}
\pysigstopsignatures\begin{quote}\begin{description}
\item[{Parameters}] \leavevmode
\sphinxAtStartPar
\sphinxstyleliteralstrong{\sphinxupquote{x}} (\sphinxstyleliteralemphasis{\sphinxupquote{tensor}}) \textendash{} Input tensor for the neural network

\item[{Returns}] \leavevmode
\sphinxAtStartPar
\sphinxstylestrong{l3} \textendash{} Output tensor with score for classes

\item[{Return type}] \leavevmode
\sphinxAtStartPar
tensor

\end{description}\end{quote}

\end{fulllineitems}

\index{predict() (models.FCN4.model method)@\spxentry{predict()}\spxextra{models.FCN4.model method}}

\begin{fulllineitems}
\phantomsection\label{\detokenize{models:models.FCN4.model.predict}}
\pysigstartsignatures
\pysiglinewithargsret{\sphinxbfcode{\sphinxupquote{predict}}}{\emph{\DUrole{n}{x}}}{}
\pysigstopsignatures\begin{quote}\begin{description}
\item[{Parameters}] \leavevmode
\sphinxAtStartPar
\sphinxstyleliteralstrong{\sphinxupquote{x}} (\sphinxstyleliteralemphasis{\sphinxupquote{tensor}}) \textendash{} Input tensor for the neural network

\item[{Returns}] \leavevmode
\sphinxAtStartPar
\sphinxstylestrong{labels} \textendash{} The predicted labels are returned in shape {[}n\_batches{]}.

\item[{Return type}] \leavevmode
\sphinxAtStartPar
torch.tensor

\end{description}\end{quote}

\end{fulllineitems}

\index{prob() (models.FCN4.model method)@\spxentry{prob()}\spxextra{models.FCN4.model method}}

\begin{fulllineitems}
\phantomsection\label{\detokenize{models:models.FCN4.model.prob}}
\pysigstartsignatures
\pysiglinewithargsret{\sphinxbfcode{\sphinxupquote{prob}}}{\emph{\DUrole{n}{x}}}{}
\pysigstopsignatures\begin{quote}\begin{description}
\item[{Parameters}] \leavevmode
\sphinxAtStartPar
\sphinxstyleliteralstrong{\sphinxupquote{x}} (\sphinxstyleliteralemphasis{\sphinxupquote{tensor}}) \textendash{} Input tensor for the neural network

\item[{Returns}] \leavevmode
\sphinxAtStartPar
The predicted probabilities for each class are returned in shape {[}n\_batches,n\_classes{]}.

\item[{Return type}] \leavevmode
\sphinxAtStartPar
torch.tensor

\end{description}\end{quote}

\end{fulllineitems}

\index{training (models.FCN4.model attribute)@\spxentry{training}\spxextra{models.FCN4.model attribute}}

\begin{fulllineitems}
\phantomsection\label{\detokenize{models:models.FCN4.model.training}}
\pysigstartsignatures
\pysigline{\sphinxbfcode{\sphinxupquote{training}}\sphinxbfcode{\sphinxupquote{\DUrole{p}{:}\DUrole{w}{  }bool}}}
\pysigstopsignatures
\end{fulllineitems}


\end{fulllineitems}



\section{models.base\_model module}
\label{\detokenize{models:module-models.base_model}}\label{\detokenize{models:models-base-model-module}}\index{module@\spxentry{module}!models.base\_model@\spxentry{models.base\_model}}\index{models.base\_model@\spxentry{models.base\_model}!module@\spxentry{module}}\index{PokerModelBase (class in models.base\_model)@\spxentry{PokerModelBase}\spxextra{class in models.base\_model}}

\begin{fulllineitems}
\phantomsection\label{\detokenize{models:models.base_model.PokerModelBase}}
\pysigstartsignatures
\pysigline{\sphinxbfcode{\sphinxupquote{class\DUrole{w}{  }}}\sphinxcode{\sphinxupquote{models.base\_model.}}\sphinxbfcode{\sphinxupquote{PokerModelBase}}}
\pysigstopsignatures
\sphinxAtStartPar
Bases: \sphinxcode{\sphinxupquote{torch.nn.modules.module.Module}}

\sphinxAtStartPar
Abstract base class for Image classifier
PokerModelBase provide the interfaces definition for image classifiers.
By the abstract base class, the implementation details of different models
are hidden to the agent.
\index{forward() (models.base\_model.PokerModelBase method)@\spxentry{forward()}\spxextra{models.base\_model.PokerModelBase method}}

\begin{fulllineitems}
\phantomsection\label{\detokenize{models:models.base_model.PokerModelBase.forward}}
\pysigstartsignatures
\pysiglinewithargsret{\sphinxbfcode{\sphinxupquote{abstract\DUrole{w}{  }}}\sphinxbfcode{\sphinxupquote{forward}}}{\emph{\DUrole{n}{x}}}{}
\pysigstopsignatures\begin{quote}\begin{description}
\item[{Parameters}] \leavevmode
\sphinxAtStartPar
\sphinxstyleliteralstrong{\sphinxupquote{x}} (\sphinxstyleliteralemphasis{\sphinxupquote{tensor}}) \textendash{} Input tensor for the neural network

\item[{Return type}] \leavevmode
\sphinxAtStartPar
Defined by subclasses

\end{description}\end{quote}

\end{fulllineitems}

\index{predict() (models.base\_model.PokerModelBase method)@\spxentry{predict()}\spxextra{models.base\_model.PokerModelBase method}}

\begin{fulllineitems}
\phantomsection\label{\detokenize{models:models.base_model.PokerModelBase.predict}}
\pysigstartsignatures
\pysiglinewithargsret{\sphinxbfcode{\sphinxupquote{abstract\DUrole{w}{  }}}\sphinxbfcode{\sphinxupquote{predict}}}{\emph{\DUrole{n}{x}}}{}
\pysigstopsignatures\begin{quote}\begin{description}
\item[{Parameters}] \leavevmode
\sphinxAtStartPar
\sphinxstyleliteralstrong{\sphinxupquote{x}} (\sphinxstyleliteralemphasis{\sphinxupquote{tensor}}) \textendash{} Input tensor for the neural network

\item[{Returns}] \leavevmode
\sphinxAtStartPar
The predicted labels are returned in shape {[}n\_batches{]}.

\item[{Return type}] \leavevmode
\sphinxAtStartPar
torch.tensor

\end{description}\end{quote}

\end{fulllineitems}

\index{prob() (models.base\_model.PokerModelBase method)@\spxentry{prob()}\spxextra{models.base\_model.PokerModelBase method}}

\begin{fulllineitems}
\phantomsection\label{\detokenize{models:models.base_model.PokerModelBase.prob}}
\pysigstartsignatures
\pysiglinewithargsret{\sphinxbfcode{\sphinxupquote{abstract\DUrole{w}{  }}}\sphinxbfcode{\sphinxupquote{prob}}}{\emph{\DUrole{n}{x}}}{}
\pysigstopsignatures\begin{quote}\begin{description}
\item[{Parameters}] \leavevmode
\sphinxAtStartPar
\sphinxstyleliteralstrong{\sphinxupquote{x}} (\sphinxstyleliteralemphasis{\sphinxupquote{tensor}}) \textendash{} Input tensor for the neural network

\item[{Returns}] \leavevmode
\sphinxAtStartPar
The predicted probabilities for each class are returned in shape {[}n\_batches,n\_classes{]}.

\item[{Return type}] \leavevmode
\sphinxAtStartPar
torch.tensor

\end{description}\end{quote}

\end{fulllineitems}

\index{training (models.base\_model.PokerModelBase attribute)@\spxentry{training}\spxextra{models.base\_model.PokerModelBase attribute}}

\begin{fulllineitems}
\phantomsection\label{\detokenize{models:models.base_model.PokerModelBase.training}}
\pysigstartsignatures
\pysigline{\sphinxbfcode{\sphinxupquote{training}}\sphinxbfcode{\sphinxupquote{\DUrole{p}{:}\DUrole{w}{  }bool}}}
\pysigstopsignatures
\end{fulllineitems}


\end{fulllineitems}



\section{Module contents}
\label{\detokenize{models:module-models}}\label{\detokenize{models:module-contents}}\index{module@\spxentry{module}!models@\spxentry{models}}\index{models@\spxentry{models}!module@\spxentry{module}}
\sphinxstepscope


\chapter{strategy package}
\label{\detokenize{strategy:strategy-package}}\label{\detokenize{strategy::doc}}

\section{Strategy handling:}
\label{\detokenize{strategy:strategy-handling}}
\sphinxAtStartPar
We used Counterfactual Regret Minimization(CFR) for deciding which action we would like to use. The CFR has ability to explore every possible result of the action we made and reach Nash equilibrium. Since there does not exist any strategy that can guarantee we could win every single game, using the strategy that can reach Nash equilibrium becomes a good option. The advantage of using the Nash equilibrium strategies is that our exploitability is minimum. Therefore, our agent will not totally malfunction even though the opponent knows our strategy or we gave sufficient exploitative power. Since kuhn poker is a simple poker game, the whole result can be easily calculated by algorithm. We just directly used the result from CFR and used the random seed, which is an uniform distribution, to decide which action we want to take. For the PokerBot, we will engage two types of games, 3 cards and 4 cards, we used CFR sample code from the internet and modified it to generate 3 cards and 4 cards result. The reference website as following url: \sphinxurl{https://justinsermeno.com/posts/cfr/}


\section{Submodules}
\label{\detokenize{strategy:submodules}}

\section{strategy.agent\_strategy module}
\label{\detokenize{strategy:module-strategy.agent_strategy}}\label{\detokenize{strategy:strategy-agent-strategy-module}}\index{module@\spxentry{module}!strategy.agent\_strategy@\spxentry{strategy.agent\_strategy}}\index{strategy.agent\_strategy@\spxentry{strategy.agent\_strategy}!module@\spxentry{module}}\index{PokerStrategy (class in strategy.agent\_strategy)@\spxentry{PokerStrategy}\spxextra{class in strategy.agent\_strategy}}

\begin{fulllineitems}
\phantomsection\label{\detokenize{strategy:strategy.agent_strategy.PokerStrategy}}
\pysigstartsignatures
\pysigline{\sphinxbfcode{\sphinxupquote{class\DUrole{w}{  }}}\sphinxcode{\sphinxupquote{strategy.agent\_strategy.}}\sphinxbfcode{\sphinxupquote{PokerStrategy}}}
\pysigstopsignatures
\sphinxAtStartPar
Bases: {\hyperref[\detokenize{strategy:strategy.base_strategy.StrategyBase}]{\sphinxcrossref{\sphinxcode{\sphinxupquote{strategy.base\_strategy.StrategyBase}}}}}

\sphinxAtStartPar
CFR strategy implementation
PokerStrategy uses the probabilistic results of actions generated by CFR algorithm
to determine what action the poker agent should take each time for a given game type.
\index{card3strategy() (strategy.agent\_strategy.PokerStrategy method)@\spxentry{card3strategy()}\spxextra{strategy.agent\_strategy.PokerStrategy method}}

\begin{fulllineitems}
\phantomsection\label{\detokenize{strategy:strategy.agent_strategy.PokerStrategy.card3strategy}}
\pysigstartsignatures
\pysiglinewithargsret{\sphinxbfcode{\sphinxupquote{card3strategy}}}{}{{ $\rightarrow$ str}}
\pysigstopsignatures
\sphinxAtStartPar
Using random seed to decide which action we would do. The probability range is the result from CFR algorithm.

\begin{sphinxadmonition}{note}{Note:}
\sphinxAtStartPar
The CFR result3 cards:
\begin{quote}
\begin{description}
\item[{Player1 strategy:}] \leavevmode
\sphinxAtStartPar
card    history    {[}“CHECK”, “BET”{]}J       {[}“”{]}       {[}   0.79,  0.21{]}J       {[}“CB”{]}     {[}   1.00,  0.00{]}Q       {[}“”{]}       {[}   1.00,  0.00{]}Q       {[}“CB”{]}     {[}   0.45,  0.55{]}K       {[}“”{]}       {[}   0.39,  0.61{]}K       {[}“CB”{]}     {[}   0.00,  1.00{]}

\item[{Player2 strategy:}] \leavevmode
\sphinxAtStartPar
card    history    {[}“CHECK”, “BET”{]}J       {[}“B”{]}      {[}   1.00,  0.00{]}J       {[}“C”{]}      {[}   0.67,  0.33{]}Q       {[}“B”{]}      {[}   0.66,  0.34{]}Q       {[}“C”{]}      {[}   1.00,  0.00{]}K       {[}“B”{]}      {[}   0.00,  1.00{]}K       {[}“C”{]}      {[}   0.00,  1.00{]}

\end{description}
\end{quote}
\end{sphinxadmonition}

\end{fulllineitems}

\index{card4strategy() (strategy.agent\_strategy.PokerStrategy method)@\spxentry{card4strategy()}\spxextra{strategy.agent\_strategy.PokerStrategy method}}

\begin{fulllineitems}
\phantomsection\label{\detokenize{strategy:strategy.agent_strategy.PokerStrategy.card4strategy}}
\pysigstartsignatures
\pysiglinewithargsret{\sphinxbfcode{\sphinxupquote{card4strategy}}}{}{{ $\rightarrow$ str}}
\pysigstopsignatures
\sphinxAtStartPar
Using random seed to decide which action we would do. The probability range is the result from CFR algorithm.

\begin{sphinxadmonition}{note}{Note:}
\sphinxAtStartPar
The CFR result4 cards:
\begin{quote}
\begin{description}
\item[{Player1 strategy:}] \leavevmode
\sphinxAtStartPar
card    history    {[}“CHECK”, “BET”{]}J       {[}“”{]}       {[}   0.75,  0.25{]}J       {[}“CB”{]}     {[}   1.00,  0.00{]}Q       {[}“”{]}       {[}   1.00,  0.00{]}Q       {[}“CB”{]}     {[}   0.75,  0.25{]}K       {[}“”{]}       {[}   1.00,  0.00{]}K       {[}“CB”{]}     {[}   0.00,  1.00{]}A       {[}“”{]}       {[}   0.25,  0.75{]}A       {[}“CB”{]}     {[}   0.00,  1.00{]}

\item[{Player2 strategy:}] \leavevmode
\sphinxAtStartPar
card    history    {[}“CHECK”, “BET”{]}J       {[}“B”{]}      {[}   1.00,  0.00{]}J       {[}“C”{]}      {[}   0.51,  0.49{]}Q       {[}“B”{]}      {[}   0.79,  0.21{]}Q       {[}“C”{]}      {[}   1.00,  0.00{]}K       {[}“B”{]}      {[}   0.20,  0.80{]}K       {[}“C”{]}      {[}   0.51,  0.49{]}A       {[}“B”{]}      {[}   0.00,  1.00{]}A       {[}“C”{]}      {[}   0.00,  1.00{]}

\end{description}
\end{quote}
\end{sphinxadmonition}

\end{fulllineitems}

\index{get\_strategy() (strategy.agent\_strategy.PokerStrategy method)@\spxentry{get\_strategy()}\spxextra{strategy.agent\_strategy.PokerStrategy method}}

\begin{fulllineitems}
\phantomsection\label{\detokenize{strategy:strategy.agent_strategy.PokerStrategy.get_strategy}}
\pysigstartsignatures
\pysiglinewithargsret{\sphinxbfcode{\sphinxupquote{get\_strategy}}}{}{{ $\rightarrow$ str}}
\pysigstopsignatures
\end{fulllineitems}


\end{fulllineitems}



\section{strategy.base\_strategy module}
\label{\detokenize{strategy:module-strategy.base_strategy}}\label{\detokenize{strategy:strategy-base-strategy-module}}\index{module@\spxentry{module}!strategy.base\_strategy@\spxentry{strategy.base\_strategy}}\index{strategy.base\_strategy@\spxentry{strategy.base\_strategy}!module@\spxentry{module}}\index{StrategyBase (class in strategy.base\_strategy)@\spxentry{StrategyBase}\spxextra{class in strategy.base\_strategy}}

\begin{fulllineitems}
\phantomsection\label{\detokenize{strategy:strategy.base_strategy.StrategyBase}}
\pysigstartsignatures
\pysigline{\sphinxbfcode{\sphinxupquote{class\DUrole{w}{  }}}\sphinxcode{\sphinxupquote{strategy.base\_strategy.}}\sphinxbfcode{\sphinxupquote{StrategyBase}}}
\pysigstopsignatures
\sphinxAtStartPar
Bases: \sphinxcode{\sphinxupquote{object}}

\sphinxAtStartPar
Base class for game strategy
StrategyBase provides the interface definitions for game strategy,
and defines the set value functions for inheritance.
\index{get\_strategy() (strategy.base\_strategy.StrategyBase method)@\spxentry{get\_strategy()}\spxextra{strategy.base\_strategy.StrategyBase method}}

\begin{fulllineitems}
\phantomsection\label{\detokenize{strategy:strategy.base_strategy.StrategyBase.get_strategy}}
\pysigstartsignatures
\pysiglinewithargsret{\sphinxbfcode{\sphinxupquote{get\_strategy}}}{}{}
\pysigstopsignatures
\end{fulllineitems}

\index{set\_avaliable\_actions() (strategy.base\_strategy.StrategyBase method)@\spxentry{set\_avaliable\_actions()}\spxextra{strategy.base\_strategy.StrategyBase method}}

\begin{fulllineitems}
\phantomsection\label{\detokenize{strategy:strategy.base_strategy.StrategyBase.set_avaliable_actions}}
\pysigstartsignatures
\pysiglinewithargsret{\sphinxbfcode{\sphinxupquote{set\_avaliable\_actions}}}{\emph{\DUrole{n}{avaliable\_actions}\DUrole{p}{:}\DUrole{w}{  }\DUrole{n}{list}}}{}
\pysigstopsignatures
\end{fulllineitems}

\index{set\_bank() (strategy.base\_strategy.StrategyBase method)@\spxentry{set\_bank()}\spxextra{strategy.base\_strategy.StrategyBase method}}

\begin{fulllineitems}
\phantomsection\label{\detokenize{strategy:strategy.base_strategy.StrategyBase.set_bank}}
\pysigstartsignatures
\pysiglinewithargsret{\sphinxbfcode{\sphinxupquote{set\_bank}}}{\emph{\DUrole{n}{bank}\DUrole{p}{:}\DUrole{w}{  }\DUrole{n}{int}}}{}
\pysigstopsignatures
\end{fulllineitems}

\index{set\_current\_card() (strategy.base\_strategy.StrategyBase method)@\spxentry{set\_current\_card()}\spxextra{strategy.base\_strategy.StrategyBase method}}

\begin{fulllineitems}
\phantomsection\label{\detokenize{strategy:strategy.base_strategy.StrategyBase.set_current_card}}
\pysigstartsignatures
\pysiglinewithargsret{\sphinxbfcode{\sphinxupquote{set\_current\_card}}}{\emph{\DUrole{n}{current\_card}\DUrole{p}{:}\DUrole{w}{  }\DUrole{n}{str}}}{}
\pysigstopsignatures
\end{fulllineitems}

\index{set\_gametype() (strategy.base\_strategy.StrategyBase method)@\spxentry{set\_gametype()}\spxextra{strategy.base\_strategy.StrategyBase method}}

\begin{fulllineitems}
\phantomsection\label{\detokenize{strategy:strategy.base_strategy.StrategyBase.set_gametype}}
\pysigstartsignatures
\pysiglinewithargsret{\sphinxbfcode{\sphinxupquote{set\_gametype}}}{\emph{\DUrole{n}{gametype}\DUrole{p}{:}\DUrole{w}{  }\DUrole{n}{int}}}{}
\pysigstopsignatures
\end{fulllineitems}

\index{set\_move\_history() (strategy.base\_strategy.StrategyBase method)@\spxentry{set\_move\_history()}\spxextra{strategy.base\_strategy.StrategyBase method}}

\begin{fulllineitems}
\phantomsection\label{\detokenize{strategy:strategy.base_strategy.StrategyBase.set_move_history}}
\pysigstartsignatures
\pysiglinewithargsret{\sphinxbfcode{\sphinxupquote{set\_move\_history}}}{\emph{\DUrole{n}{move\_history}\DUrole{p}{:}\DUrole{w}{  }\DUrole{n}{list}}}{}
\pysigstopsignatures
\end{fulllineitems}

\index{set\_order() (strategy.base\_strategy.StrategyBase method)@\spxentry{set\_order()}\spxextra{strategy.base\_strategy.StrategyBase method}}

\begin{fulllineitems}
\phantomsection\label{\detokenize{strategy:strategy.base_strategy.StrategyBase.set_order}}
\pysigstartsignatures
\pysiglinewithargsret{\sphinxbfcode{\sphinxupquote{set\_order}}}{\emph{\DUrole{n}{order}\DUrole{p}{:}\DUrole{w}{  }\DUrole{n}{int}}}{}
\pysigstopsignatures
\end{fulllineitems}

\index{set\_roll\_the\_dice() (strategy.base\_strategy.StrategyBase method)@\spxentry{set\_roll\_the\_dice()}\spxextra{strategy.base\_strategy.StrategyBase method}}

\begin{fulllineitems}
\phantomsection\label{\detokenize{strategy:strategy.base_strategy.StrategyBase.set_roll_the_dice}}
\pysigstartsignatures
\pysiglinewithargsret{\sphinxbfcode{\sphinxupquote{set\_roll\_the\_dice}}}{\emph{\DUrole{n}{RollTheDice}\DUrole{p}{:}\DUrole{w}{  }\DUrole{n}{bool}}}{}
\pysigstopsignatures
\end{fulllineitems}


\end{fulllineitems}



\section{Module contents}
\label{\detokenize{strategy:module-strategy}}\label{\detokenize{strategy:module-contents}}\index{module@\spxentry{module}!strategy@\spxentry{strategy}}\index{strategy@\spxentry{strategy}!module@\spxentry{module}}
\sphinxstepscope


\chapter{data\_sets module}
\label{\detokenize{data_sets:data-sets-module}}\label{\detokenize{data_sets::doc}}

\section{Data handling}
\label{\detokenize{data_sets:data-handling}}
\sphinxAtStartPar
For data handling, we generated noisy images with different rotate angle and noise level for later training. After generated data we extracted features from the images which aimed to remove the noise as much as possible. Because our noise followed uniform distribution so most of the noise intensity was bigger than 0(the pure black) so we regarded those pixels as background.

\phantomsection\label{\detokenize{data_sets:module-data_sets}}\index{module@\spxentry{module}!data\_sets@\spxentry{data\_sets}}\index{data\_sets@\spxentry{data\_sets}!module@\spxentry{module}}\index{extract\_features() (in module data\_sets)@\spxentry{extract\_features()}\spxextra{in module data\_sets}}

\begin{fulllineitems}
\phantomsection\label{\detokenize{data_sets:data_sets.extract_features}}
\pysigstartsignatures
\pysiglinewithargsret{\sphinxcode{\sphinxupquote{data\_sets.}}\sphinxbfcode{\sphinxupquote{extract\_features}}}{\emph{\DUrole{n}{img: \textless{}module \textquotesingle{}PIL.Image\textquotesingle{} from \textquotesingle{}/home/xianbo/miniconda3/lib/python3.9/site\sphinxhyphen{}packages/PIL/Image.py\textquotesingle{}\textgreater{}}}}{}
\pysigstopsignatures
\sphinxAtStartPar
Convert an image to features that serve as input to the image classifier.
\begin{quote}\begin{description}
\item[{Parameters}] \leavevmode
\sphinxAtStartPar
\sphinxstyleliteralstrong{\sphinxupquote{img}} (\sphinxstyleliteralemphasis{\sphinxupquote{Image}}) \textendash{} Image to convert to features.

\item[{Returns}] \leavevmode
\sphinxAtStartPar
\sphinxstylestrong{features} \textendash{} Extracted features in a format that can be used in the image classifier.

\item[{Return type}] \leavevmode
\sphinxAtStartPar
list/matrix/structure of int, int between zero and one

\end{description}\end{quote}

\end{fulllineitems}

\index{generate\_data\_set() (in module data\_sets)@\spxentry{generate\_data\_set()}\spxextra{in module data\_sets}}

\begin{fulllineitems}
\phantomsection\label{\detokenize{data_sets:data_sets.generate_data_set}}
\pysigstartsignatures
\pysiglinewithargsret{\sphinxcode{\sphinxupquote{data\_sets.}}\sphinxbfcode{\sphinxupquote{generate\_data\_set}}}{\emph{\DUrole{n}{n\_samples}\DUrole{p}{:}\DUrole{w}{  }\DUrole{n}{int}}, \emph{\DUrole{n}{data\_dir}\DUrole{p}{:}\DUrole{w}{  }\DUrole{n}{str}}, \emph{\DUrole{n}{num\_label}\DUrole{p}{:}\DUrole{w}{  }\DUrole{n}{int}\DUrole{w}{  }\DUrole{o}{=}\DUrole{w}{  }\DUrole{default_value}{4}}, \emph{\DUrole{n}{noise\_level}\DUrole{p}{:}\DUrole{w}{  }\DUrole{n}{float}\DUrole{w}{  }\DUrole{o}{=}\DUrole{w}{  }\DUrole{default_value}{0.2}}}{{ $\rightarrow$ None}}
\pysigstopsignatures
\sphinxAtStartPar
Generate n\_samples noisy images by using generate\_noisy\_image(), and store them in data\_dir.
\begin{quote}\begin{description}
\item[{Parameters}] \leavevmode\begin{itemize}
\item {} 
\sphinxAtStartPar
\sphinxstyleliteralstrong{\sphinxupquote{n\_samples}} (\sphinxstyleliteralemphasis{\sphinxupquote{int}}) \textendash{} Number of train/test examples to generate

\item {} 
\sphinxAtStartPar
\sphinxstyleliteralstrong{\sphinxupquote{data\_dir}} (\sphinxstyleliteralemphasis{\sphinxupquote{str}}) \textendash{} Directory for storing images. \sphinxtitleref{TRAINING\_IMAGE\_DIR}, \sphinxtitleref{TEST\_IMAGE\_DIR} are predefined path for training and testing.

\item {} 
\sphinxAtStartPar
\sphinxstyleliteralstrong{\sphinxupquote{num\_label}} (\sphinxstyleliteralemphasis{\sphinxupquote{int in}}\sphinxstyleliteralemphasis{\sphinxupquote{ {[}}}\sphinxstyleliteralemphasis{\sphinxupquote{3}}\sphinxstyleliteralemphasis{\sphinxupquote{,}}\sphinxstyleliteralemphasis{\sphinxupquote{4}}\sphinxstyleliteralemphasis{\sphinxupquote{{]}}}\sphinxstyleliteralemphasis{\sphinxupquote{, }}\sphinxstyleliteralemphasis{\sphinxupquote{default: 4}}) \textendash{} Number of unique labels to generate. First ‘num\_label’ lables in predefined LABELS will be used.

\item {} 
\sphinxAtStartPar
\sphinxstyleliteralstrong{\sphinxupquote{noise\_level}} (\sphinxstyleliteralemphasis{\sphinxupquote{flat in range}}\sphinxstyleliteralemphasis{\sphinxupquote{ {[}}}\sphinxstyleliteralemphasis{\sphinxupquote{0}}\sphinxstyleliteralemphasis{\sphinxupquote{,}}\sphinxstyleliteralemphasis{\sphinxupquote{1}}\sphinxstyleliteralemphasis{\sphinxupquote{{]}}}\sphinxstyleliteralemphasis{\sphinxupquote{, }}\sphinxstyleliteralemphasis{\sphinxupquote{default: 0.2}}) \textendash{} Probability with which a given pixel is randomized.

\end{itemize}

\end{description}\end{quote}
\subsubsection*{Examples}

\begin{sphinxVerbatim}[commandchars=\\\{\}]
\PYG{g+gp}{\PYGZgt{}\PYGZgt{}\PYGZgt{} }\PYG{n}{generate\PYGZus{}data\PYGZus{}set}\PYG{p}{(}\PYG{l+m+mi}{30}\PYG{p}{,}\PYG{n}{TRAINING\PYGZus{}IMAGE\PYGZus{}DIR}\PYG{p}{,}\PYG{l+m+mi}{4}\PYG{p}{,}\PYG{l+m+mf}{0.2}\PYG{p}{)}
\PYG{g+go}{30 pictures will be saved to TRAINING\PYGZus{}IMAGE\PYGZus{}DIR}
\end{sphinxVerbatim}

\end{fulllineitems}

\index{generate\_noisy\_image() (in module data\_sets)@\spxentry{generate\_noisy\_image()}\spxextra{in module data\_sets}}

\begin{fulllineitems}
\phantomsection\label{\detokenize{data_sets:data_sets.generate_noisy_image}}
\pysigstartsignatures
\pysiglinewithargsret{\sphinxcode{\sphinxupquote{data\_sets.}}\sphinxbfcode{\sphinxupquote{generate\_noisy\_image}}}{\emph{\DUrole{n}{rank}}, \emph{\DUrole{n}{noise\_level}}}{}
\pysigstopsignatures
\sphinxAtStartPar
Generate a noisy image with a given noise corruption. This implementation mirrors how the server generates the
images. However the exact server settings for noise\_level and ROTATE\_MAX\_ANGLE are unknown.
For the PokerBot assignment you won’t need to update this function, but remember to test it.
\begin{quote}\begin{description}
\item[{Parameters}] \leavevmode\begin{itemize}
\item {} 
\sphinxAtStartPar
\sphinxstyleliteralstrong{\sphinxupquote{rank}} (\sphinxstyleliteralemphasis{\sphinxupquote{str in}}\sphinxstyleliteralemphasis{\sphinxupquote{ {[}}}\sphinxstyleliteralemphasis{\sphinxupquote{\textquotesingle{}J\textquotesingle{}}}\sphinxstyleliteralemphasis{\sphinxupquote{, }}\sphinxstyleliteralemphasis{\sphinxupquote{\textquotesingle{}Q\textquotesingle{}}}\sphinxstyleliteralemphasis{\sphinxupquote{, }}\sphinxstyleliteralemphasis{\sphinxupquote{\textquotesingle{}K\textquotesingle{}}}\sphinxstyleliteralemphasis{\sphinxupquote{,}}\sphinxstyleliteralemphasis{\sphinxupquote{\textquotesingle{}A\textquotesingle{}}}\sphinxstyleliteralemphasis{\sphinxupquote{{]}}}) \textendash{} Original card rank.

\item {} 
\sphinxAtStartPar
\sphinxstyleliteralstrong{\sphinxupquote{noise\_level}} (\sphinxstyleliteralemphasis{\sphinxupquote{int between zero and one}}) \textendash{} Probability with which a given pixel is randomized.

\end{itemize}

\item[{Returns}] \leavevmode
\sphinxAtStartPar
\sphinxstylestrong{noisy\_img} \textendash{} A noisy image representation of the card rank.

\item[{Return type}] \leavevmode
\sphinxAtStartPar
Image

\end{description}\end{quote}
\subsubsection*{Examples}

\begin{sphinxVerbatim}[commandchars=\\\{\}]
\PYG{g+gp}{\PYGZgt{}\PYGZgt{}\PYGZgt{} }\PYG{n}{generate\PYGZus{}noisy\PYGZus{}image}\PYG{p}{(}\PYG{l+s+s2}{\PYGZdq{}}\PYG{l+s+s2}{J}\PYG{l+s+s2}{\PYGZdq{}}\PYG{p}{,}\PYG{l+m+mf}{0.2}\PYG{p}{)}
\end{sphinxVerbatim}

\end{fulllineitems}

\index{load\_data\_set() (in module data\_sets)@\spxentry{load\_data\_set()}\spxextra{in module data\_sets}}

\begin{fulllineitems}
\phantomsection\label{\detokenize{data_sets:data_sets.load_data_set}}
\pysigstartsignatures
\pysiglinewithargsret{\sphinxcode{\sphinxupquote{data\_sets.}}\sphinxbfcode{\sphinxupquote{load\_data\_set}}}{\emph{\DUrole{n}{data\_dir}}, \emph{\DUrole{n}{n\_validation}}}{}
\pysigstopsignatures
\sphinxAtStartPar
Prepare features for the images in data\_dir and divide in a training and validation set.
\begin{quote}\begin{description}
\item[{Parameters}] \leavevmode\begin{itemize}
\item {} 
\sphinxAtStartPar
\sphinxstyleliteralstrong{\sphinxupquote{data\_dir}} (\sphinxstyleliteralemphasis{\sphinxupquote{str}}) \textendash{} Directory of images to load

\item {} 
\sphinxAtStartPar
\sphinxstyleliteralstrong{\sphinxupquote{n\_validation}} (\sphinxstyleliteralemphasis{\sphinxupquote{int}}) \textendash{} Number of images that are assigned to the validation set

\end{itemize}

\item[{Returns}] \leavevmode
\sphinxAtStartPar
\begin{itemize}
\item {} 
\sphinxAtStartPar
\sphinxstylestrong{training\_features} (\sphinxstyleemphasis{list}) \textendash{} Containing the arrays of int between 0 and 1 which are the training images features.

\item {} 
\sphinxAtStartPar
\sphinxstylestrong{training\_labels} (\sphinxstyleemphasis{list}) \textendash{} Containing the training labels of str which contain alphabet among’ J,Q,K,A’.

\item {} 
\sphinxAtStartPar
\sphinxstylestrong{validation\_features} (\sphinxstyleemphasis{list}) \textendash{} Containing the arrays of int between 0 and 1 which are the validation images features.

\item {} 
\sphinxAtStartPar
\sphinxstylestrong{validation\_labels} (\sphinxstyleemphasis{list}) \textendash{} Containing the validation labels of str which contain alphabet among’ J,Q,K,A’.

\end{itemize}


\end{description}\end{quote}

\end{fulllineitems}


\sphinxstepscope


\chapter{client package}
\label{\detokenize{client:client-package}}\label{\detokenize{client::doc}}
\sphinxAtStartPar
This part is provided from the teacher. So, we don’t add much information about this part.


\section{Submodules}
\label{\detokenize{client:submodules}}

\section{client.controller module}
\label{\detokenize{client:client-controller-module}}

\section{client.events module}
\label{\detokenize{client:module-client.events}}\label{\detokenize{client:client-events-module}}\index{module@\spxentry{module}!client.events@\spxentry{client.events}}\index{client.events@\spxentry{client.events}!module@\spxentry{module}}\index{ClientRequestEventsIterator (class in client.events)@\spxentry{ClientRequestEventsIterator}\spxextra{class in client.events}}

\begin{fulllineitems}
\phantomsection\label{\detokenize{client:client.events.ClientRequestEventsIterator}}
\pysigstartsignatures
\pysigline{\sphinxbfcode{\sphinxupquote{class\DUrole{w}{  }}}\sphinxcode{\sphinxupquote{client.events.}}\sphinxbfcode{\sphinxupquote{ClientRequestEventsIterator}}}
\pysigstopsignatures
\sphinxAtStartPar
Bases: \sphinxcode{\sphinxupquote{object}}
\index{close() (client.events.ClientRequestEventsIterator method)@\spxentry{close()}\spxextra{client.events.ClientRequestEventsIterator method}}

\begin{fulllineitems}
\phantomsection\label{\detokenize{client:client.events.ClientRequestEventsIterator.close}}
\pysigstartsignatures
\pysiglinewithargsret{\sphinxbfcode{\sphinxupquote{close}}}{}{}
\pysigstopsignatures
\end{fulllineitems}

\index{is\_closed() (client.events.ClientRequestEventsIterator method)@\spxentry{is\_closed()}\spxextra{client.events.ClientRequestEventsIterator method}}

\begin{fulllineitems}
\phantomsection\label{\detokenize{client:client.events.ClientRequestEventsIterator.is_closed}}
\pysigstartsignatures
\pysiglinewithargsret{\sphinxbfcode{\sphinxupquote{is\_closed}}}{}{}
\pysigstopsignatures
\end{fulllineitems}

\index{make\_request() (client.events.ClientRequestEventsIterator method)@\spxentry{make\_request()}\spxextra{client.events.ClientRequestEventsIterator method}}

\begin{fulllineitems}
\phantomsection\label{\detokenize{client:client.events.ClientRequestEventsIterator.make_request}}
\pysigstartsignatures
\pysiglinewithargsret{\sphinxbfcode{\sphinxupquote{make\_request}}}{\emph{\DUrole{n}{request}}}{}
\pysigstopsignatures
\end{fulllineitems}

\index{next() (client.events.ClientRequestEventsIterator method)@\spxentry{next()}\spxextra{client.events.ClientRequestEventsIterator method}}

\begin{fulllineitems}
\phantomsection\label{\detokenize{client:client.events.ClientRequestEventsIterator.next}}
\pysigstartsignatures
\pysiglinewithargsret{\sphinxbfcode{\sphinxupquote{next}}}{}{}
\pysigstopsignatures
\end{fulllineitems}

\index{set\_initial\_request() (client.events.ClientRequestEventsIterator method)@\spxentry{set\_initial\_request()}\spxextra{client.events.ClientRequestEventsIterator method}}

\begin{fulllineitems}
\phantomsection\label{\detokenize{client:client.events.ClientRequestEventsIterator.set_initial_request}}
\pysigstartsignatures
\pysiglinewithargsret{\sphinxbfcode{\sphinxupquote{set\_initial\_request}}}{\emph{\DUrole{n}{request}}}{}
\pysigstopsignatures
\end{fulllineitems}


\end{fulllineitems}



\section{client.state module}
\label{\detokenize{client:module-client.state}}\label{\detokenize{client:client-state-module}}\index{module@\spxentry{module}!client.state@\spxentry{client.state}}\index{client.state@\spxentry{client.state}!module@\spxentry{module}}\index{ClientGameRoundState (class in client.state)@\spxentry{ClientGameRoundState}\spxextra{class in client.state}}

\begin{fulllineitems}
\phantomsection\label{\detokenize{client:client.state.ClientGameRoundState}}
\pysigstartsignatures
\pysiglinewithargsret{\sphinxbfcode{\sphinxupquote{class\DUrole{w}{  }}}\sphinxcode{\sphinxupquote{client.state.}}\sphinxbfcode{\sphinxupquote{ClientGameRoundState}}}{\emph{\DUrole{n}{coordinator\_id}}, \emph{\DUrole{n}{round\_id}}}{}
\pysigstopsignatures
\sphinxAtStartPar
Bases: \sphinxcode{\sphinxupquote{object}}

\sphinxAtStartPar
ClientGameRoundState tracks the state of the current round, from deal to showdown. Attributes should be accessed
through their corresponding getter and setter methods. For the PokerBot assignment you should not modify the setter
methods yourself (only test them).
\index{\_coordinator\_id (client.state.ClientGameRoundState attribute)@\spxentry{\_coordinator\_id}\spxextra{client.state.ClientGameRoundState attribute}}

\begin{fulllineitems}
\phantomsection\label{\detokenize{client:client.state.ClientGameRoundState._coordinator_id}}
\pysigstartsignatures
\pysigline{\sphinxbfcode{\sphinxupquote{\_coordinator\_id}}}
\pysigstopsignatures
\sphinxAtStartPar
Unique game coordinator identifier (token), duplicate from ClientGameState.\_coordinator\_id
\begin{quote}\begin{description}
\item[{Type}] \leavevmode
\sphinxAtStartPar
str

\end{description}\end{quote}

\end{fulllineitems}

\index{\_round\_id (client.state.ClientGameRoundState attribute)@\spxentry{\_round\_id}\spxextra{client.state.ClientGameRoundState attribute}}

\begin{fulllineitems}
\phantomsection\label{\detokenize{client:client.state.ClientGameRoundState._round_id}}
\pysigstartsignatures
\pysigline{\sphinxbfcode{\sphinxupquote{\_round\_id}}}
\pysigstopsignatures
\sphinxAtStartPar
Round counter, starts from 1
\begin{quote}\begin{description}
\item[{Type}] \leavevmode
\sphinxAtStartPar
int

\end{description}\end{quote}

\end{fulllineitems}

\index{\_card (client.state.ClientGameRoundState attribute)@\spxentry{\_card}\spxextra{client.state.ClientGameRoundState attribute}}

\begin{fulllineitems}
\phantomsection\label{\detokenize{client:client.state.ClientGameRoundState._card}}
\pysigstartsignatures
\pysigline{\sphinxbfcode{\sphinxupquote{\_card}}}
\pysigstopsignatures
\sphinxAtStartPar
Current card in hand; ‘?’ means the exact card rank is unknown and has to be recognized from \_card\_image
\begin{quote}\begin{description}
\item[{Type}] \leavevmode
\sphinxAtStartPar
str, in {[}‘J’, ‘Q’, ‘K’, ‘?’{]}

\end{description}\end{quote}

\end{fulllineitems}

\index{\_card\_image (client.state.ClientGameRoundState attribute)@\spxentry{\_card\_image}\spxextra{client.state.ClientGameRoundState attribute}}

\begin{fulllineitems}
\phantomsection\label{\detokenize{client:client.state.ClientGameRoundState._card_image}}
\pysigstartsignatures
\pysigline{\sphinxbfcode{\sphinxupquote{\_card\_image}}}
\pysigstopsignatures
\sphinxAtStartPar
Current card image in hand
\begin{quote}\begin{description}
\item[{Type}] \leavevmode
\sphinxAtStartPar
Image

\end{description}\end{quote}

\end{fulllineitems}

\index{\_turn\_order (client.state.ClientGameRoundState attribute)@\spxentry{\_turn\_order}\spxextra{client.state.ClientGameRoundState attribute}}

\begin{fulllineitems}
\phantomsection\label{\detokenize{client:client.state.ClientGameRoundState._turn_order}}
\pysigstartsignatures
\pysigline{\sphinxbfcode{\sphinxupquote{\_turn\_order}}}
\pysigstopsignatures
\sphinxAtStartPar
Player turn position for the current round, player ‘1’ acts first
\begin{quote}\begin{description}
\item[{Type}] \leavevmode
\sphinxAtStartPar
int, in {[} 1, 2 {]}

\end{description}\end{quote}

\end{fulllineitems}

\index{\_moves\_history (client.state.ClientGameRoundState attribute)@\spxentry{\_moves\_history}\spxextra{client.state.ClientGameRoundState attribute}}

\begin{fulllineitems}
\phantomsection\label{\detokenize{client:client.state.ClientGameRoundState._moves_history}}
\pysigstartsignatures
\pysigline{\sphinxbfcode{\sphinxupquote{\_moves\_history}}}
\pysigstopsignatures
\sphinxAtStartPar
Previously made actions of both players. Actions in the list alternate between players, i.e., the first element
is the first action of player ‘1’, and the second element is the first action of player ‘2’, etc. The last
element of \_moves\_history is the last action made by your opponent. If you’re the first to move, \_moves\_history
will be empty.
\begin{quote}\begin{description}
\item[{Type}] \leavevmode
\sphinxAtStartPar
list of str

\end{description}\end{quote}

\end{fulllineitems}

\index{\_available\_actions (client.state.ClientGameRoundState attribute)@\spxentry{\_available\_actions}\spxextra{client.state.ClientGameRoundState attribute}}

\begin{fulllineitems}
\phantomsection\label{\detokenize{client:client.state.ClientGameRoundState._available_actions}}
\pysigstartsignatures
\pysigline{\sphinxbfcode{\sphinxupquote{\_available\_actions}}}
\pysigstopsignatures
\sphinxAtStartPar
Available actions this turn, e.g., on the first move, \_available\_actions = {[}‘BET’, ‘CHECK’, ‘FOLD’{]}.
\begin{quote}\begin{description}
\item[{Type}] \leavevmode
\sphinxAtStartPar
list of str, where str in subset of {[}‘BET’, ‘CHECK’, ‘FOLD’, ‘CALL’{]}

\end{description}\end{quote}

\end{fulllineitems}

\index{\_outcome (client.state.ClientGameRoundState attribute)@\spxentry{\_outcome}\spxextra{client.state.ClientGameRoundState attribute}}

\begin{fulllineitems}
\phantomsection\label{\detokenize{client:client.state.ClientGameRoundState._outcome}}
\pysigstartsignatures
\pysigline{\sphinxbfcode{\sphinxupquote{\_outcome}}}
\pysigstopsignatures
\sphinxAtStartPar
Amount of chips won this round. Negative values indicate a loss.
\begin{quote}\begin{description}
\item[{Type}] \leavevmode
\sphinxAtStartPar
str

\end{description}\end{quote}

\end{fulllineitems}

\index{\_cards (client.state.ClientGameRoundState attribute)@\spxentry{\_cards}\spxextra{client.state.ClientGameRoundState attribute}}

\begin{fulllineitems}
\phantomsection\label{\detokenize{client:client.state.ClientGameRoundState._cards}}
\pysigstartsignatures
\pysigline{\sphinxbfcode{\sphinxupquote{\_cards}}}
\pysigstopsignatures
\sphinxAtStartPar
Cards at showdown for both players, concatenated in player order. I.e., ‘KJ’ indicates player ‘1’ holds a ‘K’,
and player ‘2’ holds a ‘J’. If the opposing player folds, a question\sphinxhyphen{}mark is returned for that player’s card;
i.e. ‘K?’ indicates the card for player ‘2’ was not revealed at showdown.
\begin{quote}\begin{description}
\item[{Type}] \leavevmode
\sphinxAtStartPar
str

\end{description}\end{quote}

\end{fulllineitems}

\index{add\_move\_history() (client.state.ClientGameRoundState method)@\spxentry{add\_move\_history()}\spxextra{client.state.ClientGameRoundState method}}

\begin{fulllineitems}
\phantomsection\label{\detokenize{client:client.state.ClientGameRoundState.add_move_history}}
\pysigstartsignatures
\pysiglinewithargsret{\sphinxbfcode{\sphinxupquote{add\_move\_history}}}{\emph{\DUrole{n}{move}}}{}
\pysigstopsignatures
\end{fulllineitems}

\index{get\_available\_actions() (client.state.ClientGameRoundState method)@\spxentry{get\_available\_actions()}\spxextra{client.state.ClientGameRoundState method}}

\begin{fulllineitems}
\phantomsection\label{\detokenize{client:client.state.ClientGameRoundState.get_available_actions}}
\pysigstartsignatures
\pysiglinewithargsret{\sphinxbfcode{\sphinxupquote{get\_available\_actions}}}{}{}
\pysigstopsignatures
\end{fulllineitems}

\index{get\_card() (client.state.ClientGameRoundState method)@\spxentry{get\_card()}\spxextra{client.state.ClientGameRoundState method}}

\begin{fulllineitems}
\phantomsection\label{\detokenize{client:client.state.ClientGameRoundState.get_card}}
\pysigstartsignatures
\pysiglinewithargsret{\sphinxbfcode{\sphinxupquote{get\_card}}}{}{}
\pysigstopsignatures
\end{fulllineitems}

\index{get\_card\_image() (client.state.ClientGameRoundState method)@\spxentry{get\_card\_image()}\spxextra{client.state.ClientGameRoundState method}}

\begin{fulllineitems}
\phantomsection\label{\detokenize{client:client.state.ClientGameRoundState.get_card_image}}
\pysigstartsignatures
\pysiglinewithargsret{\sphinxbfcode{\sphinxupquote{get\_card\_image}}}{}{}
\pysigstopsignatures
\end{fulllineitems}

\index{get\_cards() (client.state.ClientGameRoundState method)@\spxentry{get\_cards()}\spxextra{client.state.ClientGameRoundState method}}

\begin{fulllineitems}
\phantomsection\label{\detokenize{client:client.state.ClientGameRoundState.get_cards}}
\pysigstartsignatures
\pysiglinewithargsret{\sphinxbfcode{\sphinxupquote{get\_cards}}}{}{}
\pysigstopsignatures
\end{fulllineitems}

\index{get\_coordinator\_id() (client.state.ClientGameRoundState method)@\spxentry{get\_coordinator\_id()}\spxextra{client.state.ClientGameRoundState method}}

\begin{fulllineitems}
\phantomsection\label{\detokenize{client:client.state.ClientGameRoundState.get_coordinator_id}}
\pysigstartsignatures
\pysiglinewithargsret{\sphinxbfcode{\sphinxupquote{get\_coordinator\_id}}}{}{}
\pysigstopsignatures
\end{fulllineitems}

\index{get\_moves\_history() (client.state.ClientGameRoundState method)@\spxentry{get\_moves\_history()}\spxextra{client.state.ClientGameRoundState method}}

\begin{fulllineitems}
\phantomsection\label{\detokenize{client:client.state.ClientGameRoundState.get_moves_history}}
\pysigstartsignatures
\pysiglinewithargsret{\sphinxbfcode{\sphinxupquote{get\_moves\_history}}}{}{}
\pysigstopsignatures
\end{fulllineitems}

\index{get\_outcome() (client.state.ClientGameRoundState method)@\spxentry{get\_outcome()}\spxextra{client.state.ClientGameRoundState method}}

\begin{fulllineitems}
\phantomsection\label{\detokenize{client:client.state.ClientGameRoundState.get_outcome}}
\pysigstartsignatures
\pysiglinewithargsret{\sphinxbfcode{\sphinxupquote{get\_outcome}}}{}{}
\pysigstopsignatures
\end{fulllineitems}

\index{get\_round\_id() (client.state.ClientGameRoundState method)@\spxentry{get\_round\_id()}\spxextra{client.state.ClientGameRoundState method}}

\begin{fulllineitems}
\phantomsection\label{\detokenize{client:client.state.ClientGameRoundState.get_round_id}}
\pysigstartsignatures
\pysiglinewithargsret{\sphinxbfcode{\sphinxupquote{get\_round\_id}}}{}{}
\pysigstopsignatures
\end{fulllineitems}

\index{get\_turn\_order() (client.state.ClientGameRoundState method)@\spxentry{get\_turn\_order()}\spxextra{client.state.ClientGameRoundState method}}

\begin{fulllineitems}
\phantomsection\label{\detokenize{client:client.state.ClientGameRoundState.get_turn_order}}
\pysigstartsignatures
\pysiglinewithargsret{\sphinxbfcode{\sphinxupquote{get\_turn\_order}}}{}{}
\pysigstopsignatures
\end{fulllineitems}

\index{is\_ended() (client.state.ClientGameRoundState method)@\spxentry{is\_ended()}\spxextra{client.state.ClientGameRoundState method}}

\begin{fulllineitems}
\phantomsection\label{\detokenize{client:client.state.ClientGameRoundState.is_ended}}
\pysigstartsignatures
\pysiglinewithargsret{\sphinxbfcode{\sphinxupquote{is\_ended}}}{}{}
\pysigstopsignatures
\end{fulllineitems}

\index{set\_available\_actions() (client.state.ClientGameRoundState method)@\spxentry{set\_available\_actions()}\spxextra{client.state.ClientGameRoundState method}}

\begin{fulllineitems}
\phantomsection\label{\detokenize{client:client.state.ClientGameRoundState.set_available_actions}}
\pysigstartsignatures
\pysiglinewithargsret{\sphinxbfcode{\sphinxupquote{set\_available\_actions}}}{\emph{\DUrole{n}{available\_actions}}}{}
\pysigstopsignatures
\end{fulllineitems}

\index{set\_card() (client.state.ClientGameRoundState method)@\spxentry{set\_card()}\spxextra{client.state.ClientGameRoundState method}}

\begin{fulllineitems}
\phantomsection\label{\detokenize{client:client.state.ClientGameRoundState.set_card}}
\pysigstartsignatures
\pysiglinewithargsret{\sphinxbfcode{\sphinxupquote{set\_card}}}{\emph{\DUrole{n}{card}}}{}
\pysigstopsignatures
\end{fulllineitems}

\index{set\_card\_image() (client.state.ClientGameRoundState method)@\spxentry{set\_card\_image()}\spxextra{client.state.ClientGameRoundState method}}

\begin{fulllineitems}
\phantomsection\label{\detokenize{client:client.state.ClientGameRoundState.set_card_image}}
\pysigstartsignatures
\pysiglinewithargsret{\sphinxbfcode{\sphinxupquote{set\_card\_image}}}{\emph{\DUrole{n}{card\_image}}}{}
\pysigstopsignatures
\end{fulllineitems}

\index{set\_cards() (client.state.ClientGameRoundState method)@\spxentry{set\_cards()}\spxextra{client.state.ClientGameRoundState method}}

\begin{fulllineitems}
\phantomsection\label{\detokenize{client:client.state.ClientGameRoundState.set_cards}}
\pysigstartsignatures
\pysiglinewithargsret{\sphinxbfcode{\sphinxupquote{set\_cards}}}{\emph{\DUrole{n}{cards}}}{}
\pysigstopsignatures
\end{fulllineitems}

\index{set\_moves\_history() (client.state.ClientGameRoundState method)@\spxentry{set\_moves\_history()}\spxextra{client.state.ClientGameRoundState method}}

\begin{fulllineitems}
\phantomsection\label{\detokenize{client:client.state.ClientGameRoundState.set_moves_history}}
\pysigstartsignatures
\pysiglinewithargsret{\sphinxbfcode{\sphinxupquote{set\_moves\_history}}}{\emph{\DUrole{n}{moves\_history}}}{}
\pysigstopsignatures
\end{fulllineitems}

\index{set\_outcome() (client.state.ClientGameRoundState method)@\spxentry{set\_outcome()}\spxextra{client.state.ClientGameRoundState method}}

\begin{fulllineitems}
\phantomsection\label{\detokenize{client:client.state.ClientGameRoundState.set_outcome}}
\pysigstartsignatures
\pysiglinewithargsret{\sphinxbfcode{\sphinxupquote{set\_outcome}}}{\emph{\DUrole{n}{outcome}}}{}
\pysigstopsignatures
\end{fulllineitems}

\index{set\_turn\_order() (client.state.ClientGameRoundState method)@\spxentry{set\_turn\_order()}\spxextra{client.state.ClientGameRoundState method}}

\begin{fulllineitems}
\phantomsection\label{\detokenize{client:client.state.ClientGameRoundState.set_turn_order}}
\pysigstartsignatures
\pysiglinewithargsret{\sphinxbfcode{\sphinxupquote{set\_turn\_order}}}{\emph{\DUrole{n}{order}}}{}
\pysigstopsignatures
\end{fulllineitems}


\end{fulllineitems}

\index{ClientGameState (class in client.state)@\spxentry{ClientGameState}\spxextra{class in client.state}}

\begin{fulllineitems}
\phantomsection\label{\detokenize{client:client.state.ClientGameState}}
\pysigstartsignatures
\pysiglinewithargsret{\sphinxbfcode{\sphinxupquote{class\DUrole{w}{  }}}\sphinxcode{\sphinxupquote{client.state.}}\sphinxbfcode{\sphinxupquote{ClientGameState}}}{\emph{\DUrole{n}{coordinator\_id}}, \emph{\DUrole{n}{player\_token}}, \emph{\DUrole{n}{player\_bank}}}{}
\pysigstopsignatures
\sphinxAtStartPar
Bases: \sphinxcode{\sphinxupquote{object}}

\sphinxAtStartPar
A ClientGameState object tracks a specific game between two players. A game consists of multiple rounds from deal
to showdown. Attributes should be accessed through their corresponding getter and setter methods. For the PokerBot
assignment you should not modify the setter methods yourself (only test them).
\index{\_coordinator\_id (client.state.ClientGameState attribute)@\spxentry{\_coordinator\_id}\spxextra{client.state.ClientGameState attribute}}

\begin{fulllineitems}
\phantomsection\label{\detokenize{client:client.state.ClientGameState._coordinator_id}}
\pysigstartsignatures
\pysigline{\sphinxbfcode{\sphinxupquote{\_coordinator\_id}}}
\pysigstopsignatures
\sphinxAtStartPar
Game coordinator identifier token
\begin{quote}\begin{description}
\item[{Type}] \leavevmode
\sphinxAtStartPar
str

\end{description}\end{quote}

\end{fulllineitems}

\index{\_player\_token (client.state.ClientGameState attribute)@\spxentry{\_player\_token}\spxextra{client.state.ClientGameState attribute}}

\begin{fulllineitems}
\phantomsection\label{\detokenize{client:client.state.ClientGameState._player_token}}
\pysigstartsignatures
\pysigline{\sphinxbfcode{\sphinxupquote{\_player\_token}}}
\pysigstopsignatures
\sphinxAtStartPar
Unique player identifier token
\begin{quote}\begin{description}
\item[{Type}] \leavevmode
\sphinxAtStartPar
str

\end{description}\end{quote}

\end{fulllineitems}

\index{\_player\_bank (client.state.ClientGameState attribute)@\spxentry{\_player\_bank}\spxextra{client.state.ClientGameState attribute}}

\begin{fulllineitems}
\phantomsection\label{\detokenize{client:client.state.ClientGameState._player_bank}}
\pysigstartsignatures
\pysigline{\sphinxbfcode{\sphinxupquote{\_player\_bank}}}
\pysigstopsignatures
\sphinxAtStartPar
Amount of player credit chips
\begin{quote}\begin{description}
\item[{Type}] \leavevmode
\sphinxAtStartPar
int

\end{description}\end{quote}

\end{fulllineitems}

\index{\_rounds (client.state.ClientGameState attribute)@\spxentry{\_rounds}\spxextra{client.state.ClientGameState attribute}}

\begin{fulllineitems}
\phantomsection\label{\detokenize{client:client.state.ClientGameState._rounds}}
\pysigstartsignatures
\pysigline{\sphinxbfcode{\sphinxupquote{\_rounds}}}
\pysigstopsignatures
\sphinxAtStartPar
Tracks the individual rounds played in this game
\begin{quote}\begin{description}
\item[{Type}] \leavevmode
\sphinxAtStartPar
list of ClientGameRoundState

\end{description}\end{quote}

\end{fulllineitems}

\index{get\_coordinator\_id() (client.state.ClientGameState method)@\spxentry{get\_coordinator\_id()}\spxextra{client.state.ClientGameState method}}

\begin{fulllineitems}
\phantomsection\label{\detokenize{client:client.state.ClientGameState.get_coordinator_id}}
\pysigstartsignatures
\pysiglinewithargsret{\sphinxbfcode{\sphinxupquote{get\_coordinator\_id}}}{}{}
\pysigstopsignatures
\end{fulllineitems}

\index{get\_last\_round\_state() (client.state.ClientGameState method)@\spxentry{get\_last\_round\_state()}\spxextra{client.state.ClientGameState method}}

\begin{fulllineitems}
\phantomsection\label{\detokenize{client:client.state.ClientGameState.get_last_round_state}}
\pysigstartsignatures
\pysiglinewithargsret{\sphinxbfcode{\sphinxupquote{get\_last\_round\_state}}}{}{{ $\rightarrow$ {\hyperref[\detokenize{client:client.state.ClientGameRoundState}]{\sphinxcrossref{client.state.ClientGameRoundState}}}}}
\pysigstopsignatures
\end{fulllineitems}

\index{get\_player\_bank() (client.state.ClientGameState method)@\spxentry{get\_player\_bank()}\spxextra{client.state.ClientGameState method}}

\begin{fulllineitems}
\phantomsection\label{\detokenize{client:client.state.ClientGameState.get_player_bank}}
\pysigstartsignatures
\pysiglinewithargsret{\sphinxbfcode{\sphinxupquote{get\_player\_bank}}}{}{}
\pysigstopsignatures
\end{fulllineitems}

\index{get\_player\_token() (client.state.ClientGameState method)@\spxentry{get\_player\_token()}\spxextra{client.state.ClientGameState method}}

\begin{fulllineitems}
\phantomsection\label{\detokenize{client:client.state.ClientGameState.get_player_token}}
\pysigstartsignatures
\pysiglinewithargsret{\sphinxbfcode{\sphinxupquote{get\_player\_token}}}{}{}
\pysigstopsignatures
\end{fulllineitems}

\index{get\_rounds() (client.state.ClientGameState method)@\spxentry{get\_rounds()}\spxextra{client.state.ClientGameState method}}

\begin{fulllineitems}
\phantomsection\label{\detokenize{client:client.state.ClientGameState.get_rounds}}
\pysigstartsignatures
\pysiglinewithargsret{\sphinxbfcode{\sphinxupquote{get\_rounds}}}{}{}
\pysigstopsignatures
\end{fulllineitems}

\index{start\_new\_round() (client.state.ClientGameState method)@\spxentry{start\_new\_round()}\spxextra{client.state.ClientGameState method}}

\begin{fulllineitems}
\phantomsection\label{\detokenize{client:client.state.ClientGameState.start_new_round}}
\pysigstartsignatures
\pysiglinewithargsret{\sphinxbfcode{\sphinxupquote{start\_new\_round}}}{}{}
\pysigstopsignatures
\end{fulllineitems}

\index{update\_bank() (client.state.ClientGameState method)@\spxentry{update\_bank()}\spxextra{client.state.ClientGameState method}}

\begin{fulllineitems}
\phantomsection\label{\detokenize{client:client.state.ClientGameState.update_bank}}
\pysigstartsignatures
\pysiglinewithargsret{\sphinxbfcode{\sphinxupquote{update\_bank}}}{\emph{\DUrole{n}{outcome}}}{}
\pysigstopsignatures
\end{fulllineitems}


\end{fulllineitems}



\section{Module contents}
\label{\detokenize{client:module-client}}\label{\detokenize{client:module-contents}}\index{module@\spxentry{module}!client@\spxentry{client}}\index{client@\spxentry{client}!module@\spxentry{module}}

\chapter{Indices and tables}
\label{\detokenize{index:indices-and-tables}}\begin{itemize}
\item {} 
\sphinxAtStartPar
\DUrole{xref,std,std-ref}{genindex}

\item {} 
\sphinxAtStartPar
\DUrole{xref,std,std-ref}{modindex}

\item {} 
\sphinxAtStartPar
\DUrole{xref,std,std-ref}{search}

\end{itemize}

\begin{sphinxthebibliography}{KuhnPoke}
\bibitem[KuhnPoker]{Introduction:kuhnpoker}
\sphinxAtStartPar
Kuhn poker \sphinxhyphen{} Wikipedia \sphinxurl{https://en.wikipedia.org/wiki/Kuhn\_poker}
\end{sphinxthebibliography}


\renewcommand{\indexname}{Python Module Index}
\begin{sphinxtheindex}
\let\bigletter\sphinxstyleindexlettergroup
\bigletter{a}
\item\relax\sphinxstyleindexentry{agent}\sphinxstyleindexpageref{agent:\detokenize{module-agent}}
\indexspace
\bigletter{c}
\item\relax\sphinxstyleindexentry{client}\sphinxstyleindexpageref{client:\detokenize{module-client}}
\item\relax\sphinxstyleindexentry{client.events}\sphinxstyleindexpageref{client:\detokenize{module-client.events}}
\item\relax\sphinxstyleindexentry{client.state}\sphinxstyleindexpageref{client:\detokenize{module-client.state}}
\indexspace
\bigletter{d}
\item\relax\sphinxstyleindexentry{data\_sets}\sphinxstyleindexpageref{data_sets:\detokenize{module-data_sets}}
\indexspace
\bigletter{m}
\item\relax\sphinxstyleindexentry{models}\sphinxstyleindexpageref{models:\detokenize{module-models}}
\item\relax\sphinxstyleindexentry{models.base\_model}\sphinxstyleindexpageref{models:\detokenize{module-models.base_model}}
\item\relax\sphinxstyleindexentry{models.CNN3}\sphinxstyleindexpageref{models:\detokenize{module-models.CNN3}}
\item\relax\sphinxstyleindexentry{models.CNN4}\sphinxstyleindexpageref{models:\detokenize{module-models.CNN4}}
\item\relax\sphinxstyleindexentry{models.FCN3}\sphinxstyleindexpageref{models:\detokenize{module-models.FCN3}}
\item\relax\sphinxstyleindexentry{models.FCN4}\sphinxstyleindexpageref{models:\detokenize{module-models.FCN4}}
\indexspace
\bigletter{s}
\item\relax\sphinxstyleindexentry{strategy}\sphinxstyleindexpageref{strategy:\detokenize{module-strategy}}
\item\relax\sphinxstyleindexentry{strategy.agent\_strategy}\sphinxstyleindexpageref{strategy:\detokenize{module-strategy.agent_strategy}}
\item\relax\sphinxstyleindexentry{strategy.base\_strategy}\sphinxstyleindexpageref{strategy:\detokenize{module-strategy.base_strategy}}
\end{sphinxtheindex}

\renewcommand{\indexname}{Index}
\printindex
\end{document}